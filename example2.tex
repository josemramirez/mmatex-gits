\documentclass{article}\usepackage{graphicx} \usepackage{amsmath} \usepackage{colortbl}\title{Cosmology 101 - Version 0.1}
\author{J. M. Ram{\'i}rez,$^{1}$ Co-Author1,$^{4}$ Co-Author2,$^{5}$}
\date{\today}\begin{document}
\maketitle \begin{abstract}    Behavioral responses to epidemic threats significantly impact disease transmission dynamics, yet existing epidemiological models often oversimplify these responses by relying on single-metric threshold approaches. While such models capture basic behavioral changes, they fail to reflect the nuanced decision-making processes where individuals consider multiple factors when assessing risk.
 This study presents a novel modification to the classical SEIR (Susceptible-Exposed-Infectious-Recovered) model by incorporating a dual-metric behavioral response function that simultaneously considers both current infection levels and short-term growth rates. We develop a mathematical framework where contact rate reductions are dynamically adjusted based on the interaction between absolute case numbers and their 5-day growth trajectories, hypothesizing that maximum behavioral response occurs when both metrics simultaneously indicate high risk.
 Through numerical simulations and comparative analysis, we demonstrate that this dual-metric approach more accurately captures real-world behavioral adaptation patterns compared to traditional threshold-based models. Our results show significant differences in predicted epidemic trajectories, with the dual-metric model generally predicting lower peak infection rates but potentially longer epidemic duration due to more nuanced behavioral modulation. This work provides valuable insights for public health planning and demonstrates the importance of incorporating multi-dimensional risk assessment in epidemic modeling.    \end{abstract}\section{Introduction}
The dynamics of infectious disease transmission are fundamentally shaped by human behavior, with individuals modifying their contact patterns in response to perceived epidemic risks. Classical epidemiological models have traditionally focused on biological and demographic factors while treating behavioral responses as either absent or highly simplified. However, recent experiences with COVID-19 and other emerging diseases have highlighted the critical importance of incorporating realistic behavioral adaptations into epidemic forecasting.\section{Background}
The study of epidemic dynamics through compartmental models has a rich history dating back to the seminal work of Kermack and McKendrick \cite{anderson1992infectious}. These models partition populations into distinct disease states and track their evolution through time using systems of ordinary differential equations. The SEIR model, in particular, has emerged as a fundamental framework for analyzing diseases with latent periods \cite{hethcote2000mathematics}.\end{document}