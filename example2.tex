\documentclass{article}\usepackage{graphicx} \usepackage{amsmath} \usepackage{colortbl}\title{Cosmology 101 - Version 0.1}
\author{J. M. Ram{\'i}rez,$^{1}$ Co-Author1,$^{4}$ Co-Author2,$^{5}$}
\date{\today}\begin{document}
\maketitle\begin{abstract}
In this paper, we present a comprehensive numerical investigation into the behavior of null geodesic pairs in the vicinity of a black hole horizon, aiming to elucidate the mechanisms associated with Hawking radiation. Utilizing Eddington–Finkelstein coordinates, we simulate the trajectories of paired null geodesics and quantitatively analyze their separation rates to obtain estimates of vacuum fluctuation probabilities. By generating detailed temperature profiles as a function of radius, our approach offers a novel numerical perspective on the expected thermal spectrum of Hawking radiation. This research is motivated by the inherent challenges of modeling quantum effects in curved space-times, particularly the accurate treatment of horizon physics and the extraction of observable signatures from complex dynamical systems. Our contribution lies in the development of a robust computational framework that bridges general relativistic geodesic analysis with quantum field theoretic predictions in black hole environments. The validity of our model is verified through systematic experiments that compare generated phase-space trajectory plots and extracted temperature profiles with established theoretical predictions, thereby providing significant insights into the quantum phenomena occurring near black hole horizons.
\end{abstract}\section{Introduction}

In this paper, we present a comprehensive numerical analysis of Hawking radiation by investigating the behavior of paired null geodesics in the vicinity of a black hole horizon. Our study is motivated by the challenge of reconciling quantum effects with the dynamical background of curved spacetime, a problem that has been at the forefront of theoretical physics since Hawking's seminal work \cite{Hawking1975}. We focus on the simulation of null geodesic trajectories using Eddington\textendash Finkelstein coordinates, which allows us to accurately handle horizon-crossing behavior and explore the nuances of horizon physics.

The primary objectives of our work are as follows:
\begin{itemize}
    \item To numerically simulate paired null geodesic trajectories in the background of a black hole and analyze their separation rates.
    \item To estimate vacuum fluctuation probabilities through a detailed study of the dynamics of these geodesic pairs.
    \item To generate temperature profiles as a function of the radial coordinate and compare these with theoretical predictions for Hawking radiation.
    \item To provide a robust computational framework that bridges general relativistic geodesic analysis with quantum field theoretic predictions.
\end{itemize}

This research is particularly challenging due to several factors. First, the accurate modeling of null geodesics near the event horizon requires sophisticated numerical techniques to handle the severe curvature and coordinate singularities inherent in such regimes. Second, the extraction of observables that reflect quantum phenomena from these classical trajectories necessitates a careful and precise analysis to ensure that the numerical estimates faithfully represent quantum field theoretic predictions \cite{Jacobson1993}. Finally, ensuring the stability and convergence of our numerical methods demands rigorous testing and validation against established theoretical results.

To address these challenges, our contribution lies in the development and deployment of a high-fidelity numerical framework that not only simulates the intricate behavior of paired null geodesics but also systematically extracts relevant observables that characterize Hawking radiation. In summary, our approach can be appreciated as a longer version of the abstract of the paper, and its relevance is underscored by the need for new numerical methods in the study of quantum effects in strong gravitational fields. The main contributions of this work are summarized below:
\begin{itemize}
    \item Establishment of a numerical scheme for tracing null geodesic trajectories in Eddington\textendash Finkelstein coordinates.
    \item Quantitative analysis of geodesic pair separation rates as a proxy for vacuum fluctuation probabilities.
    \item Generation of accurate temperature profiles indicative of the thermal spectrum of Hawking radiation.
    \item Verification of our model through systematic experiments that compare phase-space trajectory plots and thermal profiles with existing theoretical predictions \cite{Unruh1976}.
\end{itemize}

The rest of this paper is organized as follows. We first describe the numerical setup and simulation framework employed in our analysis. Next, we detail the results obtained from our experiments and conclude with a discussion of future work and potential extensions of this study. This investigation not only enhances our understanding of horizon physics but also paves the way for further exploration of quantum gravitational effects in strong-field regimes.\end{document}