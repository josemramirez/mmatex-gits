
% Setting up the document class for Beamer
\documentclass{beamer}

% Including essential packages
\usepackage[utf8]{inputenc}
\usepackage[T1]{fontenc}
\usepackage{lmodern}

% Configuring the Beamer theme
\usetheme{default}

% Defining the title and author
%% Libreria de formulas %%%%%%%%%%%%%%%%%%%%%%%%%%%%%%%%%%%%
%% Velocidad en el Mov. Rec. Uniforme %%%%%%%%%%%%%%
\def\vmru{
%% Eq. 1 %%%%%%%%%%%%%
\begin{equation}
v=\frac{x}{t}
\label{vmru}
\end{equation}
%%%%%%%%%%%%%%%%%%%%%%%%%%%%%%%%%%%%%%%%%%%%%%%%%%%%
}

%% Distancia en el Mov. Rec. Uniforme %%%%%%%%%%%%%%
\def\xmru{
%% Eq. 2 %%%%%%%%%%%%%
\begin{equation}
x=vt
\label{xmru}
\end{equation}
%%%%%%%%%%%%%%%%%%%%%%%%%%%%%%%%%%%%%%%%%%%%%%%%%%%%
}

%% Tiempo en el Mov. Rec. Uniforme %%%%%%%%%%%%%%%%%
\def\tmru{
%% Eq. 3 %%%%%%%%%%%%%
\begin{equation}
t=\frac{x}{v}
\label{tmru}
\end{equation}
%%%%%%%%%%%%%%%%%%%%%%%%%%%%%%%%%%%%%%%%%%%%%%%%%%%%
}

%% Libreria de Notas    %%%%%%%%%%%%%%%%%%%%%%%%%%%%%%%%%%%%
%%%%%%%%%%%%%%%%%%%%%%%%%%%%%%%%%%%%%%%%%%%%%%%%%%%%
%%%%%%%%%%% NU para la velocidad %%%%%%%%%%%%%%%%%%%
%%%%%%%%%%%%%%%%%%%%%%%%%%%%%%%%%%%%%%%%%%%%%%%%%%%%
\def\vNUa%
{
%% Dibuja una cajita!! %%%%%%%%%%%%%%%%%%%
%% el background de gris..
\psframebox[linewidth=2pt,framearc=.3,fillstyle=solid,
fillcolor=lightgray]{
\begin{tabular}{c}
%%%%%%%%%%%%%%%%%%%%%%%%%%%%%%%%%%%%%%%%%%
{\bf Notas de Unidades}\\
{\sc magnitud}: Velocidad [v].
La velocidad puede venir en: \\

$\frac{{\rm km}}{{\rm h}}$ ; 
$\frac{{\rm m}}{{\rm s}}$ ; 
$\frac{{\rm cm}}{{\rm s}}$. \\

$1~\frac{{\rm km}}{{\rm h}}=
\frac{1000}{3600}~\frac{{\rm m}}{{\rm s}}
=\cancelto{0.277}{\frac{1}{3.6}}\frac{{\rm m}}{{\rm s}}$ \\

$1~\frac{{\rm m}}{{\rm s}} = 100~\frac{{\rm cm}}{{\rm s}}$

%%%%%%%%%%%%%%%%%%%%%%%%%%%%%%%%%%%%%%%%%%
\end{tabular}
}
%%%%%%%%%%%%%%%%%%%%%%%%%%%%%%%%%%%%%%%%%%

%% fin definicion de vNUa
}
%%%%%%%%%%%%%%%%%%%%%%%%%%%%%%%%%%%%%%%%%%%%%%%%%%%%
%%%%%%%%%%%%%%%%%%%%%%%%%%%%%%%%%%%%%%%%%%%%%%%%%%%%
%%%%%%%%%%%%%%%%%%%%%%%%%%%%%%%%%%%%%%%%%%%%%%%%%%%%

%%%%%%%%%%%%%%%%%%%%%%%%%%%%%%%%%%%%%%%%%%%%%%%%%%%%%%%%%%%%

%% Titulo y autor .. obligatorio..
\title{MRU i1: ver1}
\subtitle{Practica... y las puertas se abren}
\author{ejjr}


\date{\today}

\begin{document}

% Creating the title slide
\begin{frame}
    \titlepage
\end{frame}

% Creating a simple content slide
\begin{frame}{Título de la Diapositiva}
    \begin{itemize}
        \item Punto 1: Este es un ejemplo sencillo.
        \item Punto 2: Usamos Beamer en LaTeX.
        \item Punto 3: ¡Personaliza como quieras!
    \end{itemize}
\end{frame}



%________________________________________________________________(ult. col, 65)
% ========================== slide1 ============================= 
\begin{frame}{MRU}
% 
% -Paragraph1 
%----------------------------------------------------------------
%----------------------------------------------------------------
% -Idea1 
% ============================================================== 
 
Un << auto >> { anda } a { 152 km/h }. Comienza su pase por un puente y { 205 segundos } despu\'es est\'a en el otro extremo, saliendo completamente (del puente) { 15 segundos } despu\'es. ?`Cu\'al es la { longitud } del puente y cu\'al es la longitud del << auto >>? %% Hay 0 elementos de distancia %% Hay 2 elementos de tiempo %% Hay 1 elementos de velocidad


% ============================================================== 
% -Idea2 
% ============================================================== 
 
\begin{itemize}                                                              

\item El 1er paso en la resoluci\'on de cualquier problema, es la
identificaci\'on de los elementos del problema.            
              
\end{itemize}

% ============================================================== 
%----------------------------------------------------------------
%----------------------------------------------------------------
\end{frame}

\end{document}
