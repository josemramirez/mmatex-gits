\documentclass{article}
\usepackage{graphicx}
 \usepackage{amsmath}
 \usepackage[utf8]{inputenc}
 \usepackage[T1]{fontenc}
 \usepackage{hyperref}
 \usepackage{url}
 \usepackage{booktabs}
 \usepackage{amsfonts}
 \usepackage{nicefrac}
 \usepackage{microtype}
 \usepackage{titletoc}
 \usepackage{subcaption}
  \usepackage{multirow}
 \usepackage{color}
 \usepackage{colortbl}
 \usepackage{cleveref}
 \usepackage{algorithm}
 \usepackage{algorithmicx}
 \usepackage{algpseudocode}
 \graphicspath{{../}}
 \DeclareMathOperator*{\argmin}{arg\,min}
 \DeclareMathOperator*{\argmax}{arg\,max}


\title{Cosmology 101 - Version 0.1}
\author{J. M. Ram{\'i}rez,$^{1}$ Co-Author1,$^{4}$ Co-Author2,$^{5}$}
\date{\today}
\newcommand{\fix}{\marginpar{FIX}}
\newcommand{\new}{\marginpar{NEW}}\begin{document}
\maketitle \begin{abstract}
We present an interactive visualization tool that leverages Eddington-Finkelstein coordinates to map out the light cone structure in the vicinity of a black hole horizon. Our work aims to elucidate the deformation of light paths and causal connections as they approach and traverse the event horizon, offering a dynamic perspective that accommodates multiple reference frames. Understanding the intricate causal structure near black holes is both conceptually challenging and computationally intensive due to the non-linear distortions imposed by intense gravitational fields. To address these difficulties, we have developed a suite of 3D plots that seamlessly integrate real-time data visualization with advanced coordinate transformations, thereby providing an intuitive view of horizon crossing phenomena. Experimental results, including comparative analyses across various reference frames and parameter sensitivity studies, substantiate our approach and shed light on the underpinnings of relativistic horizon dynamics, contributing to both theoretical insights and educational tools in gravitational physics.\end{abstract}\section{Introduction}
\section{Introduction}In recent decades, understanding the causal structure in the vicinity of black hole horizons has emerged as a central challenge in gravitational physics. Black holes, with their intense gravitational fields, impose non-linear distortions on spacetime that make the visualization and interpretation of light paths highly non-trivial. In this work, we present an interactive visualization tool that leverages Eddington-Finkelstein coordinates to map out the structure of light cones near the event horizon, thereby offering new insights into horizon crossing phenomena. Our goal is to provide a dynamic perspective that accommodates multiple reference frames, which is essential in exploring the deformation of light cones and the altered causal connections as one approaches and traverses the event horizon \cite{Einstein1916, Hawking1973}.Understanding the intricate structure of such extreme environments is both conceptually challenging and computationally intensive. The severe curvature near a black hole not only complicates the mathematical modeling but also limits the efficacy of traditional static visualizations. To surmount these difficulties, our approach integrates advanced coordinate transformations with real-time data visualization, offering a comprehensive tool for both theoretical exploration and educational purposes.Our contributions can be summarized as follows:\begin{itemize}    \item We develop a novel real-time 3D visualization suite that renders the evolution of light cone structures in the curved spacetime around black holes.    \item We employ Eddington-Finkelstein coordinates to accurately capture the causal deformation near the event horizon, thereby overcoming some of the limitations imposed by conventional coordinate systems.    \item We design the interface to accommodate multiple reference frames, which facilitates comparative studies and enhances the interpretability of relativistic horizon dynamics.    \item We verify our methodology through rigorous experimental validation, including parameter sensitivity studies and comparative analyses across different observational frames \cite{Penrose1965, Misner1973}.\end{itemize}The remainder of this paper details the methodology behind the interactive tool and the mathematical framework that underpins our visualization technique. Our work not only advances the current understanding of black hole causal structure but also provides a valuable resource for both researchers and educators in the field of gravitational physics. Furthermore, the framework presented here opens avenues for future research in the interactive visualization of other complex relativistic systems.\section{Background}
\section{Background}In this section, we place our work in the context of previous advances in gravitational physics and elucidate the formal problem setting underlying our approach. We build upon classical studies and the conceptual frameworks developed in \cite{Einstein1916, Hawking1973, Penrose1965, Misner1973} to address the deformations of light cones in the vicinity of black hole horizons.\subsection{Academic Ancestors and Conceptual Foundations}The causal structure of spacetime around black holes has long been a subject of intense study. Early work by Einstein \cite{Einstein1916} and subsequent developments by Hawking \cite{Hawking1973} laid the foundations for understanding how gravitational fields shape the propagation of light. Further contributions by Penrose \cite{Penrose1965} and Misner \cite{Misner1973} advanced the theoretical framework necessary to describe complex phenomena such as horizon crossing and singularity formation. These seminal works introduced innovative methods for analyzing causal connections, including the use of conformal and Penrose diagrams, which have been instrumental in visualizing the non-trivial topology of black hole spacetimes.\subsection{Problem Setting and Formalism}The core problem addressed in this work is the visualization of the deformation of light cones in the vicinity of a black hole horizon. Our approach utilizes Eddington-Finkelstein coordinates to overcome the coordinate singularities present in the Schwarzschild metric. The transformation to these coordinates is given by\begin{equation}v = t + r^*, \quad \text{with} \quad r^* = r + 2M \ln\left|\frac{r}{2M} - 1\right|, \end{equation}where $t$ denotes the Schwarzschild time coordinate, $r$ the radial coordinate, and $M$ the mass of the black hole. This transformation provides a smooth extension through the event horizon by eliminating the artificial singularity at $r = 2M$.In our analysis, we assume that the spacetime is both stationary and spherically symmetric and that the black hole is non-rotating. These assumptions simplify the dynamics by reducing the system to one where the interplay between gravitational time dilation and spatial curvature can be isolated and studied without the additional complexity introduced by angular momentum. Furthermore, we posit that the coordinate transformation employed is invertible and that the Lorentzian signature of the metric is maintained throughout the region of interest.By integrating these established theoretical constructs with our novel interactive visualization techniques, we provide a powerful tool for exploring horizon crossing phenomena. Our visualization suite combines real-time data rendering with advanced coordinate transformations to deliver an intuitive representation of the dynamic causal structure near black holes, thereby extending both the pedagogical and research-oriented applications of these classical results.\section{Method}
\section{Method}In accordance with the formalism introduced in the Problem Setting and building on the conceptual foundations described in the Background section \cite{Einstein1916, Hawking1973, Penrose1965, Misner1973}, we develop an interactive visualization tool that maps out the dynamic evolution of the light cone structure near the event horizon of a non-rotating black hole. Our methodology is designed to provide both an intuitive understanding of relativistic horizon dynamics and a rigorous computational framework for simulating light cone deformations in extreme gravitational fields.\subsection{Coordinate Transformation}To facilitate a smooth characterization of the causal structure across the event horizon, we adopt the Eddington-Finkelstein coordinate transformation. This is achieved by transforming the Schwarzschild coordinates using\begin{equation}v = t + r^*, \quad \text{with} \quad r^* = r + 2M\ln\left|\frac{r}{2M} - 1\right|, \end{equation}where $t$ is the Schwarzschild time coordinate, $r$ the radial coordinate, and $M$ the mass of the black hole. The transformation ensures that the metric remains regular at $r = 2M$, thereby eliminating the coordinate singularity and allowing for a continuous description of horizon crossing phenomena.\subsection{Interactive Visualization Framework}Following the coordinate transformation, we employ a suite of 3D plots for real-time data visualization. The key components of our interactive visualization framework include:\begin{itemize}    \item \textbf{Dynamic Rendering:} The visualization engine continuously renders the evolution of light cones based on the transformed coordinates, providing a real-time depiction of how light paths deform in the vicinity of the black hole horizon.    \item \textbf{Multiple Reference Frames:} By accommodating multiple reference frames, our tool enables users to compare how different observers perceive the horizon crossing event. Each frame is computed using the same underlying coordinate transformation, ensuring consistency across views.    \item \textbf{Parameter Sensitivity:} The interface allows for dynamic adjustments of parameters such as the black hole mass $M$, facilitating detailed sensitivity studies on the causal structure as a function of gravitational intensity.\end{itemize}\subsection{Computational Methodology}The computational procedure underlying our visualization tool comprises several key steps:\begin{enumerate}    \item \textbf{Pre-computation of Coordinate Grids:} Coordinate grids are generated in the Eddington-Finkelstein space for a comprehensive range of $r$ values, extending from regions outside the event horizon to regions within it.    \item \textbf{Numerical Simulation of Light Ray Propagation:} Using numerical integration techniques, the trajectories of light rays are computed within the curved spacetime geometry. This step captures the intricate interplay between gravitational time dilation and spatial curvature.    \item \textbf{Integration with the Rendering Engine:} The simulation data is fed into a 3D rendering engine that generates synchronized visualizations across different reference frames, thus producing a coherent and interactive depiction of the evolving light cone structure.    \item \textbf{Optimization for Real-Time Performance:} To support a fluid interactive experience, parallel computing methods are employed to optimize the rendering pipeline and ensure rapid data updates.\end{enumerate}In summary, our method integrates the theoretical insights from \cite{Einstein1916, Hawking1973, Penrose1965, Misner1973} with advanced computational techniques to deliver an interactive visualization tool. This tool not only enhances our understanding of horizon crossing phenomena by dynamically portraying the causal structure near black holes but also serves as an educational framework for exploring the nuances of relativistic gravitational dynamics.\end{document}