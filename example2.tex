
%% Beginning of file == mru.tex ==
%% 
%% Created 20-07-2025
\documentclass[%
%%%%%%%%%%%%%%%%%%%%
%% Final o draft? %%
final,
%draft,
%%%%%%%%%%%%%%%%%%%%
%% Total o no toal?%
total,
%nototal,
%%%%%%%%%%%%%%%%%%%%
%% Color o BW?%%%%%%
%slideColor,
slideBW,
%%%%%%%%%%%%%%%%%%%%
%% colorBG o no? %%%
%nocolorBG,
colorBG,
%%%%%%%%%%%%%%%%%%%%
%% PS o PDF %%%%%%%%
%ps,
pdf,
%%%%%%%%%%%%%%%%%%%%
%% otra opcion.. %%%
accumulate,
%noaccumulate,
%%%%%%%%%%%%%%%%%%%%
%%%%%%%%%%%%%%%%%%%%
%% Background %%%%%%
%frames
%azure
%contemporain
%nuancegris
%troispoints,
%lignesbleues
%darkblue
%alienglow
%autumn
%%%%%%%%%%%%%%%%%%%%
%%%%%%%%%%%%%%%%%%%%
]{prosper}

%%%%%%%%%%%%%%%%%%%%%%%%%%%%%%%%%%%%%%%%%%%%%%%%%%%%%%%%%%%%
%% Paquetes.. %%%%%%%%%%%%%%%%%%%%%%%%%%%%%%%%%%%%%%%%%%%%%
\usepackage[latin1]{inputenc}
\usepackage{pstricks,pstricks-add,pst-node,pst-text,pst-3d}
\usepackage{amsmath}
\usepackage{cancel}
%% Mate
\usepackage{pst-func}
\usepackage{pst-math}

%%%%%%%%%%%%%%%%%%%%%%%%%%%%%%%%%%%%%
%% Seguridad.. %%%%%%%%%%%%%%%%%%%%%%
\hypersetup{
pdfpagemode={FullScreen},
pdftitle={MRU i1: ver1},
pdfauthor={JM~Ramirez},
pdfcreator={JM~Ramirez},
pdfproducer={JM~Ramirez},
pdfkeywords={Movimiento,Rectilineo,Uniforme}}
%%%%%%%%%%%%%%%%%%%%%%%%%%%%%%%%%%%%%
%%%%%%%%%%%%%%%%%%%%%%%%%%%%%%%%%%%%%

%% You can insert definicions here..

%% DEFINICIONES MIAS..
%\def\psPiFour{12.566371}
%\def\psPiTwo{6.283185}
\def\psPi{3.14159265}
%\def\psPiH{1.570796327}
%\newdimen\pstRadUnit
%\newdimen\pstRadUnitInv
%\pstRadUnit=1.047198cm % this is pi/3
%\pstRadUnitInv=0.95493cm % this is 3/pi


%% Libreria de formulas %%%%%%%%%%%%%%%%%%%%%%%%%%%%%%%%%%%%
%% Velocidad en el Mov. Rec. Uniforme %%%%%%%%%%%%%%
\def\vmru{
%% Eq. 1 %%%%%%%%%%%%%
\begin{equation}
v=\frac{x}{t}
\label{vmru}
\end{equation}
%%%%%%%%%%%%%%%%%%%%%%%%%%%%%%%%%%%%%%%%%%%%%%%%%%%%
}

%% Distancia en el Mov. Rec. Uniforme %%%%%%%%%%%%%%
\def\xmru{
%% Eq. 2 %%%%%%%%%%%%%
\begin{equation}
x=vt
\label{xmru}
\end{equation}
%%%%%%%%%%%%%%%%%%%%%%%%%%%%%%%%%%%%%%%%%%%%%%%%%%%%
}

%% Tiempo en el Mov. Rec. Uniforme %%%%%%%%%%%%%%%%%
\def\tmru{
%% Eq. 3 %%%%%%%%%%%%%
\begin{equation}
t=\frac{x}{v}
\label{tmru}
\end{equation}
%%%%%%%%%%%%%%%%%%%%%%%%%%%%%%%%%%%%%%%%%%%%%%%%%%%%
}

%% Libreria de Notas    %%%%%%%%%%%%%%%%%%%%%%%%%%%%%%%%%%%%
%%%%%%%%%%%%%%%%%%%%%%%%%%%%%%%%%%%%%%%%%%%%%%%%%%%%
%%%%%%%%%%% NU para la velocidad %%%%%%%%%%%%%%%%%%%
%%%%%%%%%%%%%%%%%%%%%%%%%%%%%%%%%%%%%%%%%%%%%%%%%%%%
\def\vNUa%
{
%% Dibuja una cajita!! %%%%%%%%%%%%%%%%%%%
%% el background de gris..
\psframebox[linewidth=2pt,framearc=.3,fillstyle=solid,
fillcolor=lightgray]{
\begin{tabular}{c}
%%%%%%%%%%%%%%%%%%%%%%%%%%%%%%%%%%%%%%%%%%
{\bf Notas de Unidades}\\
{\sc magnitud}: Velocidad [v].
La velocidad puede venir en: \\

$\frac{{\rm km}}{{\rm h}}$ ; 
$\frac{{\rm m}}{{\rm s}}$ ; 
$\frac{{\rm cm}}{{\rm s}}$. \\

$1~\frac{{\rm km}}{{\rm h}}=
\frac{1000}{3600}~\frac{{\rm m}}{{\rm s}}
=\cancelto{0.277}{\frac{1}{3.6}}\frac{{\rm m}}{{\rm s}}$ \\

$1~\frac{{\rm m}}{{\rm s}} = 100~\frac{{\rm cm}}{{\rm s}}$

%%%%%%%%%%%%%%%%%%%%%%%%%%%%%%%%%%%%%%%%%%
\end{tabular}
}
%%%%%%%%%%%%%%%%%%%%%%%%%%%%%%%%%%%%%%%%%%

%% fin definicion de vNUa
}
%%%%%%%%%%%%%%%%%%%%%%%%%%%%%%%%%%%%%%%%%%%%%%%%%%%%
%%%%%%%%%%%%%%%%%%%%%%%%%%%%%%%%%%%%%%%%%%%%%%%%%%%%
%%%%%%%%%%%%%%%%%%%%%%%%%%%%%%%%%%%%%%%%%%%%%%%%%%%%

%%%%%%%%%%%%%%%%%%%%%%%%%%%%%%%%%%%%%%%%%%%%%%%%%%%%%%%%%%%%

%% Titulo y autor .. obligatorio..
\title{MRU i1: ver1}
\subtitle{Practica... y las puertas se abren}
\author{ejjr}
\email{ejjr.clases@gmail.com}

%%%%%%%%%%%%%%%%%%%%%%%%%%%%%%%%%%%
%%% Comienzo del documento.. %%%%%%
%%%%%%%%%%%%%%%%%%%%%%%%%%%%%%%%%%%
\begin{document}
\maketitle
%%%%%%%%%%%%%%%%%%%%%%%%%%%%%%%%%%

%%%%%%% Para pegar el log.. %%%%%%%%%%%%%%%%%%%%%%%%%%%%%%%
%%%%%%%%%%%%%%%%%%%%%%%%%%%%%%%%%%%%%%%%%%%%%%%%%%%%%%%%%%%

%%%%%%% Advertencia para no dejar que se copien %%%%%%%%%%%
\slideCaption{{\gray \large Esta nota tiene la intenci\'on de que \'este
material no sea entregado como tarea.}}
%%%%%%%%%%%%%%%%%%%%%%%%%%%%%%%%%%%%%%%%%%%%%%%%%%%%%%%%%%%

%________________________________________________________________(ult. col, 65)
% ========================== slide1 ============================= 
\begin{slide}{MRU}
% 
% -Paragraph1 
%----------------------------------------------------------------
%----------------------------------------------------------------
% -Idea1 
% ============================================================== 
 
Un << auto >> { anda } a { 152 km/h }. Comienza su pase por un puente y { 205 segundos } despu\'es est\'a en el otro extremo, saliendo completamente (del puente) { 15 segundos } despu\'es. ?`Cu\'al es la { longitud } del puente y cu\'al es la longitud del << auto >>? %% Hay 0 elementos de distancia %% Hay 2 elementos de tiempo %% Hay 1 elementos de velocidad


% ============================================================== 
% -Idea2 
% ============================================================== 
 
\begin{itemize}                                                              

\item El 1er paso en la resoluci\'on de cualquier problema, es la
identificaci\'on de los elementos del problema.            
              
\end{itemize}

% ============================================================== 
%----------------------------------------------------------------
%----------------------------------------------------------------
\end{slide}

%________________________________________________________________(ult. col, 65)
% ========================== slide1 ============================= 
\begin{slide}{MRU}
% 
% -Paragraph1 
%----------------------------------------------------------------
% ============================================================== 
% -Idea1 
% ============================================================== 
 
%% Dibuja una cajita!! %%%%%%%%%%%%%%%%%%%
%% el background de gris..
\psframebox[linewidth=2pt,framearc=.3,fillstyle=solid,
fillcolor=lightgray]{
\begin{tabular}{c}
%%%%%%%%%%%%%%%%%%%%%%%%%%%%%%%%%%%%%%%%%%

{\bf Elementos del Problema}\\
$t_{1}=205$ s  \\
$t_{2}=15$ s  \\
$t_{Total}=(t_{1}+t_{2})=220$ s  \\
$v=152$ km/h \\
$l_{\text{auto}}=?$   \\
$l_{\text{puente}}=?$ \\ 
$x_{AC}=(l_{\text{auto}}+l_{\text{puente}})=$? \\
    .. Este problema tiene un {\it truquito}! \\
    La distancia total desde (A) hasta (C) \\
    es: $x_{AC}=vt_{Total}$ \\
    Ve la figura que sigue :) 

%%%%%%%%%%%%%%%%%%%%%%%%%%%%%%%%%%%%%%%%%%
\end{tabular}
}
%%%%%%%%%%%%%%%%%%%%%%%%%%%%%%%%%%%%%%%%%%

% Generated with LaTeXDraw 2.0.0
% Fri Aug 01 20:56:48 CEST 2008
% \usepackage[usenames,dvipsnames]{pstricks}
% \usepackage{epsfig}
% \usepackage{pst-grad} % For gradients
% \usepackage{pst-plot} % For axes
\scalebox{1} % Change this value to rescale the drawing.
{
\begin{pspicture}(0,-0.3528646)(3.3785417,0.3528646)
\usefont{T1}{ptm}{m}{n}
\rput(10.2,4){\psframebox[linewidth=0.04]{Qu\'e f\'ormula uso?}}
\end{pspicture}
}

% ============================================================== 
%----------------------------------------------------------------
%----------------------------------------------------------------
\end{slide}

%________________________________________________________________(ult. col, 65)
% ========================== slide2 ============================= 
\begin{slide}{MRU}
% 
% -Paragraph1 
%----------------------------------------------------------------
%----------------------------------------------------------------
% -Idea1 
% ==============================================================

% Generated with LaTeXDraw 2.0.0
% Sat Aug 09 20:46:30 CEST 2008
% \usepackage[usenames,dvipsnames]{pstricks}
% \usepackage{epsfig}
% \usepackage{pst-grad} % For gradients
% \usepackage{pst-plot} % For axes
\scalebox{0.7} % Change this value to rescale the drawing.
{
\begin{pspicture}(0,-5.194844)(13.665937,5.194844)
\definecolor{color934}{rgb}{0.8980392156862745,0.25882352941176473,0.25882352941176473}
\psline[linewidth=0.04cm](4.720625,3.8373437)(6.620625,3.8373437)
\psline[linewidth=0.04cm](4.680625,3.9973438)(4.680625,3.6573439)
\psline[linewidth=0.04cm](6.660625,3.9773438)(6.660625,3.6973438)
\psline[linewidth=0.04cm](4.680625,3.5973437)(4.680625,3.2573438)
\psline[linewidth=0.04cm](4.680625,3.4173439)(5.080625,3.4173439)
\psline[linewidth=0.04cm](6.080625,3.4173439)(6.640625,3.4173439)
\psline[linewidth=0.04cm](6.660625,3.5773437)(6.660625,3.2773438)
\usefont{T1}{ptm}{m}{n}
\rput(5.66125,4.1473436){\color{color934}\text{auto}}
\usefont{T1}{ptm}{m}{n}
\rput(5.6607814,3.4073439){$l_{\text{auto}}$}
\usefont{T1}{ptm}{m}{n}
\rput(4.6590624,4.1873436){(A)}
\usefont{T1}{ptm}{m}{n}
\rput(6.6390624,4.1873436){(B)}
\psline[linewidth=0.04cm,arrowsize=0.05291667cm 2.0,arrowlength=1.4,arrowinset=0.4]{->}(4.700625,2.9173439)(6.700625,2.8973436)
\usefont{T1}{ptm}{m}{n}
\rput(5.674375,2.6073437){v=152 km/h}
\psline[linewidth=0.04cm](6.760625,1.8373437)(10.020625,1.8373437)
\psline[linewidth=0.04cm](6.720625,1.9973438)(6.720625,1.6573437)
\psline[linewidth=0.04cm](10.060625,1.9773438)(10.060625,1.6973437)
\psline[linewidth=0.04cm](6.720625,1.5973438)(6.720625,1.2573438)
\psline[linewidth=0.04cm](6.720625,1.4223437)(7.200625,1.4223437)
\psline[linewidth=0.04cm](8.780625,1.4373437)(10.020625,1.4223437)
\psline[linewidth=0.04cm](10.060625,1.5973438)(10.060625,1.2973437)
\usefont{T1}{ptm}{m}{n}
\rput(8.389218,2.2473438){\color{color934}pte.}
\usefont{T1}{ptm}{m}{n}
\rput(8.050781,1.4473437){$l_{\text{puente}}$}
\usefont{T1}{ptm}{m}{n}
\rput(10.0390625,2.3673437){(C)}
\usefont{T1}{ptm}{m}{n}
\rput(6.9590626,4.987344){(1) Al comienzo de la obervaci\'on.}
\usefont{T1}{ptm}{m}{n}
\rput(6.793594,-1.6126562){entre (A) y (C) es $t_{tot}=220$ s y que $x_{AC}=vt_{tot}$}
\usefont{T1}{ptm}{m}{n}
\rput(6.1648436,-1.0726563){El truco de este problema, es que el tiempo que pasa}
\psline[linewidth=0.04cm](4.680625,0.8773438)(10.020625,0.85734373)
\psline[linewidth=0.04cm](10.060625,1.0773437)(10.060625,0.6773437)
\psline[linewidth=0.04cm](4.680625,1.0573437)(4.680625,0.69734377)
\psline[linewidth=0.04cm](4.680625,0.63734376)(4.680625,0.37734374)
\psline[linewidth=0.04cm](4.720625,0.49734375)(5.160625,0.49734375)
\psline[linewidth=0.04cm](9.540625,0.47734374)(9.940625,0.47734374)
\psline[linewidth=0.04cm](10.060625,0.63734376)(10.060625,0.31734374)
\usefont{T1}{ptm}{m}{n}
\rput(7.520781,0.48734376){$x_{AC}=l_{\text{auto}}+l_{\text{puente}}$}
\pscircle[linewidth=0.04,dimen=outer](3.810625,-0.37265626){0.23}
\usefont{T1}{ptm}{m}{n}
\rput(3.82125,-0.39265624){A}
\psline[linewidth=0.04cm,arrowsize=0.05291667cm 2.0,arrowlength=1.4,arrowinset=0.4]{->}(4.000625,-0.14265625)(4.600625,0.41734374)
\pscircle[linewidth=0.04,dimen=outer](10.910625,-0.41265625){0.23}
\usefont{T1}{ptm}{m}{n}
\rput(10.905937,-0.43265626){C}
\psline[linewidth=0.04cm,arrowsize=0.05291667cm 2.0,arrowlength=1.4,arrowinset=0.4]{->}(10.640625,-0.26265624)(10.240625,0.25734374)
\psline[linewidth=0.04cm](6.180625,-2.5826561)(9.440625,-2.5826561)
\psline[linewidth=0.04cm](6.140625,-2.4226563)(6.140625,-2.7626562)
\psline[linewidth=0.04cm](9.480625,-2.4426563)(9.480625,-2.7226562)
\psline[linewidth=0.04cm](6.140625,-2.8226562)(6.140625,-3.1626563)
\psline[linewidth=0.04cm](6.140625,-2.9976563)(6.620625,-2.9976563)
\psline[linewidth=0.04cm](8.400625,-2.9826562)(9.440625,-2.9976563)
\psline[linewidth=0.04cm](9.480625,-2.8226562)(9.480625,-3.1226563)
\usefont{T1}{ptm}{m}{n}
\rput(7.809219,-2.1726563){\color{color934}pte.}
\usefont{T1}{ptm}{m}{n}
\rput(7.3507814,-2.9926562){$l_{\text{puente}}$}
\usefont{T1}{ptm}{m}{n}
\rput(9.459063,-2.0526562){(C)}
\psline[linewidth=0.04cm](7.540625,-3.7626562)(9.440625,-3.7626562)
\psline[linewidth=0.04cm](7.500625,-3.6026564)(7.500625,-3.9426563)
\psline[linewidth=0.04cm](9.480625,-3.6226563)(9.480625,-3.9026563)
\psline[linewidth=0.04cm](7.500625,-4.0026565)(7.500625,-4.342656)
\psline[linewidth=0.04cm](7.500625,-4.1826563)(7.900625,-4.1826563)
\psline[linewidth=0.04cm](8.900625,-4.1826563)(9.460625,-4.1826563)
\psline[linewidth=0.04cm](9.480625,-4.0226564)(9.480625,-4.322656)
\usefont{T1}{ptm}{m}{n}
\rput(8.48125,-3.4526563){\color{color934}\text{auto}}
\usefont{T1}{ptm}{m}{n}
\rput(8.500781,-4.192656){$l_{\text{auto}}$}
\usefont{T1}{ptm}{m}{n}
\rput(7.4790626,-3.4126563){(A)}
\usefont{T1}{ptm}{m}{n}
\rput(9.459063,-3.4126563){(B)}
\psline[linewidth=0.04cm,arrowsize=0.05291667cm 2.0,arrowlength=1.4,arrowinset=0.4]{->}(7.520625,-4.6826563)(9.520625,-4.7026563)
\usefont{T1}{ptm}{m}{n}
\rput(8.494375,-4.992656){v=152 km/h}
\end{pspicture} 
}

% ============================================================== 
%----------------------------------------------------------------
%----------------------------------------------------------------
\end{slide}

%________________________________________________________________(ult. col, 65)
% ========================== slide2 ============================= 
\begin{slide}{MRU}
% 
% -Paragraph1 
%----------------------------------------------------------------
%----------------------------------------------------------------
% -Idea1 
% ============================================================== 
 
{\huge                
\xmru
}                                             
 
% ============================================================== 
% -Idea2 
% ============================================================== 

%%%%%%%%%%%%%%%%%%%%%%%%%%%%%%%%%%%%%%%%%%%%%%%%%%%%%%%%%%%%%%%%
%%%%%%%%%%%%%%%%% Cajita de Texto sencilla %%%%%%%%%%%%%%%%%%%%%
\begin{center}
\psdblframebox[linewidth=1.5pt,linecolor=red]{%
\parbox[c]{5cm}{
1 km =  1000 metros \\
y \\
1 hora =  3600 segundos \\

}}
\end{center}
%%%%%%%%%%%%%%%%%%%%%%%%%%%%%%%%%%%%%%%%%%%%%%%%%%%%%%%%%%%%%%%%
%%%%%%%%%%%%%%%%%%%%%%%%%%%%%%%%%%%%%%%%%%%%%%%%%%%%%%%%%%%%%%%%
                           
% ============================================================== 
% -Idea3 
% ============================================================== 
 
Finalmente..                                                              
                                                               
                                                               
% ============================================================== 
%----------------------------------------------------------------
%----------------------------------------------------------------
\end{slide}

%________________________________________________________________(ult. col, 65)
% ========================== slide3 ============================= 
\begin{slide}{MRU}
% 
% -Paragraph1 
%----------------------------------------------------------------
% ============================================================== 
% -Idea1
% ==============================================================

La longitud del puente es:
{\large
\begin{equation} 
l_{\text{puente}}=
152~(\frac{1000}{3600})~
\frac{m}{\cancel{s}}~[205~\cancel{s}]=
8655~m
\end{equation}
}

                               
% ============================================================== 
% -Idea2 
% ============================================================== 

La distancia total desde (A) a (C) es: 
{\large
\begin{equation} 
x_{AC}=
152~(\frac{1000}{3600})~
\frac{m}{\cancel{s}}~[220~\cancel{s}]=
9288~m
\end{equation}
}

% ============================================================== 
%----------------------------------------------------------------
%----------------------------------------------------------------
\end{slide}

%________________________________________________________________(ult. col, 65)
% ========================== slide3 ============================= 
\begin{slide}{MRU}
% 
% -Paragraph1 
%----------------------------------------------------------------
% ============================================================== 
% -Idea1
% ==============================================================

La longitud del auto es simplemente:
{\large
\begin{equation} 
l_{\text{auto}}=
x_{AC}-l_{\text{puente}}
=
633~m
\end{equation}
}

                               
% ============================================================== 
% -Idea2 
% ============================================================== 




% ============================================================== 
%----------------------------------------------------------------
%----------------------------------------------------------------
\end{slide}


%%%%%%%%%%%%%%%%%%%%%%%%%%%%%%%%%%%%%%%%%%%%%%%%%%%%%%
%%%%%%%%%%%%%%%  Tail  %%%%%%%%%%%%%%%%%%%%%%%%%%%%%%%
%%%%%%%%%%%%%%%%%%%%%%%%%%%%%%%%%%%%%%%%%%%%%%%%%%%%%%

\end{document}
%%
%% End of file == mru.tex ==
%%
%% Modified 20-07-2025 
