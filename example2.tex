\documentclass{article}\usepackage{graphicx} \usepackage{amsmath} \usepackage{colortbl}\title{Cosmology 101 - Version 0.1}
\author{J. M. Ram{\'i}rez,$^{1}$ Co-Author1,$^{4}$ Co-Author2,$^{5}$}
\date{\today}\begin{document}
\maketitle\begin{abstract}    Behavioral responses to epidemic threats significantly impact disease transmission dynamics, yet existing epidemiological models often oversimplify these responses by relying on single-metric threshold approaches. While such models capture basic behavioral changes, they fail to reflect the nuanced decision-making processes where individuals consider multiple factors when assessing risk.         % Identifying the gap and our approach    This study presents a novel modification to the classical SEIR (Susceptible-Exposed-Infectious-Recovered) model by incorporating a dual-metric behavioral response function that simultaneously considers both current infection levels and short-term growth rates. We develop a mathematical framework where contact rate reductions are dynamically adjusted based on the interaction between absolute case numbers and their 5-day growth trajectories, hypothesizing that maximum behavioral response occurs when both metrics simultaneously indicate high risk.        % Methods and results    Through numerical simulations and comparative analysis, we demonstrate that this dual-metric approach more accurately captures real-world behavioral adaptation patterns compared to traditional threshold-based models. Our results show significant differences in predicted epidemic trajectories, with the dual-metric model generally predicting lower peak infection rates but potentially longer epidemic duration due to more nuanced behavioral modulation. This work provides valuable insights for public health planning and demonstrates the importance of incorporating multi-dimensional risk assessment in epidemic modeling.    \end{abstract}}\end{document}