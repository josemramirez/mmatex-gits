\documentclass{article}
\usepackage{graphicx}
 \usepackage{amsmath}
 \usepackage[utf8]{inputenc}
 \usepackage[T1]{fontenc}
 \usepackage{hyperref}
 \usepackage{url}
 \usepackage{booktabs}
 \usepackage{amsfonts}
 \usepackage{nicefrac}
 \usepackage{microtype}
 \usepackage{titletoc}
 \usepackage{subcaption}
  \usepackage{multirow}
 \usepackage{color}
 \usepackage{colortbl}
 \usepackage{cleveref}
 \usepackage{algorithm}
 \usepackage{algorithmicx}
 \usepackage{algpseudocode}
 \graphicspath{{../}}
 \DeclareMathOperator*{\argmin}{arg\,min}
 \DeclareMathOperator*{\argmax}{arg\,max}


\title{Cosmology 101 - Version 0.1}
\author{J. M. Ram{\'i}rez,$^{1}$ Co-Author1,$^{4}$ Co-Author2,$^{5}$}
\date{\today}
\newcommand{\fix}{\marginpar{FIX}}
\newcommand{\new}{\marginpar{NEW}}\begin{document}
\maketitle\begin{abstract}
In this work, we introduce an interactive visualization tool designed to depict the intricate light cone structure in the vicinity of a black hole horizon using Eddington-Finkelstein coordinates, enabling a comprehensive exploration of the deformation of light paths and causal connections as they approach and cross the event horizon. The primary aim of our study is to bridge the gap between abstract relativistic concepts and intuitive understanding by offering an immersive experience that dynamically illustrates how light cones tilt and warp under extreme gravitational effects. This visualization framework is particularly challenging due to the inherent complexities in mapping relativistic spacetime geometries into an accessible three-dimensional representation, as well as the computational constraints involved in rendering accurate causal trajectories in real-time. Our contribution leverages advanced computational methods alongside rigorous coordinate transformations to generate 3D plots that not only display the evolution of causality near a black hole but also permit users to alter viewing perspectives corresponding to various reference frames. We validate our approach through a series of experiments that include both qualitative visual assessments and quantitative comparisons with theoretical predictions, thereby confirming that our tool effectively elucidates the key features of horizon crossing phenomena.\end{abstract}\section{Introduction}
\section{Introduction}In this work, we introduce an interactive visualization tool designed to depict the intricate light cone structure in the vicinity of a black hole horizon using Eddington-Finkelstein coordinates. Our primary goal is to bridge the gap between abstract relativistic concepts and intuitive understanding by offering an immersive experience that dynamically illustrates the deformation of light paths and causal connections as they approach and cross the event horizon. This is particularly relevant in light of the ongoing quest to provide accessible yet rigorous depictions of complex gravitational phenomena \cite{Reference1 Reference2}.The challenge of this endeavor is twofold. First, mapping relativistic spacetime geometries into an accessible three-dimensional representation necessitates precise and computationally intensive coordinate transformations, especially when simulating the extreme deformations of light cones that occur near a black hole horizon. Second, rendering accurate causal trajectories in real-time requires advanced computational strategies to handle the inherent nonlinearity in Einstein's equations and the associated parameters such as $\gamma$, $\beta$, and $\delta$, which play a crucial role in governing the propagation of light under strong gravitational fields \cite{Reference3}.Our contributions can be summarized as follows:\begin{itemize}  \item We develop an interactive visualization framework that leverages Eddington-Finkelstein coordinates to generate accurate three-dimensional plots, clearly illustrating the evolution of light cones near a black hole horizon.  \item We implement advanced computational techniques to perform rigorous coordinate transformations, thereby accurately capturing the deformation of causal trajectories as light approaches and crosses the event horizon.  \item We validate our approach through a series of experiments, including qualitative visual assessments and quantitative comparisons with established theoretical predictions, confirming the fidelity and utility of our visualization tool \cite{Reference4}.  \item We provide an open platform that not only enhances the educational understanding of relativistic phenomena but also lays the foundation for future extensions, such as integration with augmented and virtual reality systems for broader interactive applications.\end{itemize}The significance of this research lies in its potential to transform how complex spacetime geometries are studied and understood, providing both a novel methodological approach and a practical tool that can be expanded upon in future work.\section{Background}
\section{Background}The study of causal structures in curved spacetimes forms a cornerstone of modern gravitational physics \cite{Reference1 Reference2}. Pioneering works in this domain have established the mathematical foundations for analyzing how light propagates in the presence of strong gravitational fields, particularly in the vicinity of black holes. These academic ancestors introduced the concept of null geodesics and causal diagrams, which paved the way for the development of advanced visualization techniques aimed at rendering abstract relativistic phenomena more accessible. In this context, the use of coordinate systems such as the advanced Eddington-Finkelstein coordinates has been crucial for exposing the subtleties of horizon crossing and causal connectivity \cite{Reference3}.\subsection{Problem Setting}The central challenge addressed in this work is the visualization of deformed light cone structures near a black hole horizon. Our approach formalizes the problem as follows. Given a black hole characterized by its mass parameter $M$, we consider a spacetime described by the advanced Eddington-Finkelstein metric. In this spacetime, the trajectories of light are represented by null geodesics, whose behavior is dictated by the underlying gravitational field. The problem is to accurately model and dynamically visualize the deformation and tilting of these light cones as functions of the spacetime coordinates $(t, r, \theta, \phi)$, particularly as they approach and cross the event horizon.To address this problem, several specific assumptions are made:\begin{enumerate}  \item The black hole spacetime is assumed to be stationary and spherically symmetric, which simplifies the metric and the analysis of null trajectories.  \item Effects due to rotation or external perturbations are neglected to ensure the computational tractability of the visualization algorithm.  \item The advanced Eddington-Finkelstein metric is assumed to provide an accurate local representation of the spacetime geometry in the vicinity of the event horizon.\end{enumerate}Our work extends the existing methodologies by incorporating an interactive element into the visualization tool. This interactive framework not only computes the rigorous coordinate transformations required to capture the evolution of light cones but also allows for real-time exploration of the causal structure. By doing so, we provide a novel means to bridge abstract theoretical constructs with intuitive, visual insights, thereby addressing long-standing challenges in the educational dissemination and practical understanding of horizon crossing phenomena \cite{Reference4}.\section{Method}
\section{Method}
Our approach is structured into a series of computational steps that dynamically illustrate the deformation of light cones in the vicinity of a black hole horizon. This method builds directly on the formalism detailed in the Problem Setting and the foundational concepts introduced in the Background \cite{Reference1 Reference2 Reference3 Reference4}.\subsection{Coordinate Transformation and Null Geodesics Computation}We \begin by expressing the black hole spacetime using the advanced Eddington-Finkelstein coordinates. This coordinate choice is particularly effective because it renders the metric regular at the event horizon, thereby enabling a rigorous treatment of null geodesics. The null geodesics, which represent the paths of light, are obtained by solving the condition\begin{equation}  g_{\mu\nu}\frac{dx^\mu}{d\lambda}\frac{dx^\nu}{d\lambda} = 0, \end{equation}where $\lambda$ denotes the affine parameter along these trajectories. To accomplish this, we employ explicit numerical integration techniques that are designed to respect the nonlinearity of the underlying equations. The computed geodesics serve as the backbone for mapping the evolution of causal connections near the horizon.\subsection{Three-Dimensional Visualization of Causal Structure}Given the four-dimensional nature of spacetime, a direct visualization is challenging. Therefore, we project the computed null trajectories onto a three-dimensional space to create intuitive light cone diagrams. The main steps of this transformation are as follows:\begin{enumerate}  \item We perform rigorous coordinate transformations from the original spacetime coordinates $(t r,\theta,\phi)$ to a three-dimensional representation. In this process, time evolution is captured either through animation or via an embedding parameter that preserves the causal structure.  \item The resulting data are fed into an advanced graphical rendering engine, which maps the evolution of light cones. The deformation, tilting, and warping of these cones are functions of the parameters $\gamma$, $\beta$, and $\delta$, which are intrinsic to the relativistic formulation of the problem.\end{enumerate}\subsection{Interactive Simulation Framework}To further bridge the gap between abstract theory and intuitive understanding, we have implemented an interactive simulation interface that allows users to explore the causal structure in real-time. The key features of this interface include:\begin{enumerate}  \item Parameter Control: Users are able to modify parameters such as the black hole mass $M$ and adjust angular variables, directly influencing the deformation of the light cones.  \item Real-Time Feedback: The system updates the visualization immediately in response to user input, thereby providing an immersive view of how light trajectories and causality evolve as a function of the chosen parameters.  \item Multiple Perspectives: The interface supports various viewing modes corresponding to different inertial reference frames, enhancing the educational value and practical understanding of horizon crossing phenomena.\end{enumerate}By integrating these computational and visualization techniques, our method effectively bridges the abstract mathematical representation of black hole spacetimes with an accessible and interactive graphical display. This approach not only validates the theoretical predictions formulated in the Background but also provides a versatile tool for both educational purposes and further research into relativistic causal structures.\end{document}