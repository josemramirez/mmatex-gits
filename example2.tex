\documentclass{article}\usepackage{graphicx} \usepackage{amsmath} \usepackage{colortbl}\title{Cosmology 101 - Version 0.1}
\author{J. M. Ram{\'i}rez,$^{1}$ Co-Author1,$^{4}$ Co-Author2,$^{5}$}
\date{\today}\begin{document}
\maketitle\begin{abstract} Behavioral responses to epidemic threats significantly impact disease transmission dynamics, yet existing epidemiological models often oversimplify these responses by relying on single-metric threshold approaches. While such models capture basic behavioral changes, they fail to reflect the nuanced decision-making processes where individuals consider multiple factors when assessing risk.  This study presents a novel modification to the classical SEIR (Susceptible-Exposed-Infectious-Recovered) model by incorporating a dual-metric behavioral response function that simultaneously considers both current infection levels and short-term growth rates. We develop a mathematical framework where contact rate reductions are dynamically adjusted based on the interaction between absolute case numbers and their 5-day growth trajectories, hypothesizing that maximum behavioral response occurs when both metrics simultaneously indicate high risk.  Through numerical simulations and comparative analysis, we demonstrate that this dual-metric approach more accurately captures real-world behavioral adaptation patterns compared to traditional threshold-based models. Our results show significant differences in predicted epidemic trajectories, with the dual-metric model generally predicting lower peak infection rates but potentially longer epidemic duration due to more nuanced behavioral modulation. This work provides valuable insights for public health planning and demonstrates the importance of incorporating multi-dimensional risk assessment in epidemic modeling. \end{abstract}\section{Introduction}
The dynamics of infectious disease transmission are fundamentally shaped by human behavior, with individuals modifying their contact patterns in response to perceived epidemic risks. Classical epidemiological models have traditionally focused on biological and demographic factors while treating behavioral responses as either absent or highly simplified. However, recent experiences with COVID-19 and other emerging diseases have highlighted the critical importance of incorporating realistic behavioral adaptations into epidemic forecasting.

Traditional SEIR models and their variants have served as foundational tools for understanding disease spread \cite{anderson1992infectious}. While these models have been extensively enhanced to include various biological and social factors \cite{hethcote2000mathematics}, their treatment of behavioral responses often relies on simple threshold-based approaches where populations uniformly alter contact rates once infection levels cross predetermined thresholds \cite{funk2010modelling}.

Real-world behavioral adaptation, however, exhibits far more complexity. Individuals assess multiple risk indicators when deciding whether to modify their behavior, including both current infection levels and perceived trends in disease spread \cite{wang2020impact}. The limitations of single-metric behavioral models become particularly apparent during rapidly evolving outbreak situations, where reliance solely on current case counts may fail to capture anticipatory behavioral changes triggered by concerning growth trajectories.

This paper addresses these limitations by developing an enhanced SEIR framework incorporating a dual-metric behavioral response function. Our key contributions include:

\begin{itemize}
\item Development of a novel mathematical framework combining both current infection levels and short-term growth rates to model behavioral responses
\item Implementation of a continuous response function that captures gradual behavioral adaptation rather than discrete threshold changes
\item Demonstration of improved prediction accuracy compared to traditional threshold-based approaches through extensive numerical simulations
\item Analysis of the impact of various parameter combinations on epidemic trajectories and overall disease burden
\end{itemize}

The proposed model builds upon existing work in behavioral epidemiology \cite{verelst2016behavioural} while introducing a more nuanced approach to capturing risk perception and response. By incorporating both static and dynamic risk metrics, our framework better reflects the multi-dimensional nature of individual decision-making during epidemics. This advancement is particularly relevant for public health planning and intervention design, as it provides a more realistic basis for predicting population-level behavioral changes and their subsequent impact on disease transmission.\end{document}